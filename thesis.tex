\documentclass[12pt,oneside,final]{fithesis2}
\usepackage[slovak]{babel}
\usepackage[utf8]{inputenc}
\usepackage[T1]{fontenc}
\usepackage[plainpages=false,pdfpagelabels,unicode]{hyperref}

\thesistitle{Prezentační režim pro PDF prohlížeč \emph{evince}}
\thesissubtitle{Bakalárska Práca}
\thesisstudent{Lukáš Bezdička}
\thesiswoman{false}
\thesisfaculty{fi}
\thesisyear{2011}
\thesisadvisor{RNDr. Jan Kasprzak}
\thesislang{sk}

\begin{document}
\FrontMatter
\ThesisTitlePage

\begin{ThesisDeclaration}
\DeclarationText
\AdvisorName
\end{ThesisDeclaration}

\begin{ThesisThanks}
Rád by som poďakoval vývojárom \emph{evince} za ustretovú pomoc pri písaní práce a obzvlášť José Aliste.
\end{ThesisThanks}  
 
\begin{ThesisAbstract}  
...
\end{ThesisAbstract}

\begin{ThesisKeyWords}
keyword1, keyword2, etc.
\end{ThesisKeyWords}
 
\MainMatter
%\sloppy
%\chapter{Predhovor}
\setcounter{tocdepth}{3}
\tableofcontents 
 
\chapter{Úvod}

 
\chapter{GNOME Platforma}

V roku 1997 Spencer Kimball a Peter Mattis vytvorili projekt GTK+ pre projekt GIMP. V tom istom roku Miguel de Icaza a Federico Mena založili projekt GNOME, ktorého cieľom je vytvorit jednoduché prostredie pracovnej plochy postavený na GTK+. Tieto projekty zostali na seba silne naviazané a spoločne s ďaľšími tvoria platformu GNOME. V tabuľke \ref{tab.GNOME} je prehľad GNOME platformy. Dalej sú uvedené len knižnice použité na riešenie problému.
\begin{table}[h]
\begin{center}
\begin{tabular}{|c||c||c||c||c||c||c|}
\hline \multicolumn{3}{|c||}{\begin{tiny}
\textit{Užívateľské prostredie}
\end{tiny}} & \begin{tiny}
\textit{Multimédiá}
\end{tiny} & \begin{tiny}
\textit{Komunikácia}
\end{tiny} & \begin{tiny}
\textit{Ukladanie dát}
\end{tiny} & \begin{tiny}
\textit{Utility}
\end{tiny}\\
\multicolumn {1}{|c}{GTK+} & \multicolumn {1} {c} {Cairo} & \multicolumn {1} {c||} {Clutter} & GStreamer & Telepathy & EDS & Champlain \\
\multicolumn{1}{|c}{ATK} & \multicolumn{1}{c}{Pango} & \multicolumn{1}{c||}{Webkit} & Canberra & Avahi & GDA & Enchant \\ \cline{1-3}
\multicolumn{3}{|c||}{\begin{tiny}
\textit{Jadro}
\end{tiny}} & Pulseaudio &GUPnP & Tracker & Poppler \\
\multicolumn{1}{|c}{GIO} & \multicolumn{1}{c}{GLib} & \multicolumn{1}{c||}{GOBject} & & & GeoClue & \\ \hline \hline
\multicolumn{3}{|c||}{\begin{tiny}
\textit{Systémová integrácia}
\end{tiny}} & \multicolumn{4}{|c|}{{\tiny \textit{Integrácia do pracovného prostredia}}} \\
\multicolumn{1}{|c}{upower} & \multicolumn{1}{c}{udisks} & \multicolumn{1}{c||}{policykit} & \multicolumn{1}{|c}{packagekit} & \multicolumn{1}{c}{libnotify} & \multicolumn{2}{c|}{seahorse} \\
\hline 
\end{tabular}
\caption{Platforma GNOME}
\label{tab.GNOME}
\end{center}
\end{table}

\section{GLib}
Obecná viacplatformná knižnica funkcií GLib vznikla oddelením kódu ktorý nebol priamo špecifický pre grafické užívatelské rozhranie GTK+. GLib poskytuje abstrakciu nad platformami a vytvára základ programu. GLib sa dá ďalej rozdeliť na niekoľko častí. GLib Fundamentals, kde sú základné typy a ich limity, štandardné makrá, makrá na typovú konverziu a endianitu, číselné definície matematických konštant a operácie s desatinnou čiarkou, špeciálne makrá a podporu atomických operácií. GLib Core Application Support poskytujúcu správu udalostí, podporu vlákien, asynchrónne fronty, dynamické moduly, správu pamäte, vstupno-výstupné kanály a systém správ. GLib Utilities, v ktorej sa okrem iného nachádza podpora pre časovače a funkcie na prácu s URI\footnote{Uniform Resource Identifier - jednoduchý identifikátor zdroja}. GLib Data Types na rozšírené dátové typy, napríklad n-árne stromy, cache a reťazce a GLib Tools.
O správu udalostí v programoch sa stará štruktúra GMainLoop. Tieto udalosti môžu pochádzat z rôznych zdojov ako napríklad ukazatele na súbory, časovače alebo vlastné zdroje pridané pomocou \verb|g_source_attach()|. Každá udalost má priradenú prioritu. Takisto je možné pridať funkcie, ktoré sa spustia v prípade nečinnosti.
\section{GObject}
GLib Object System je objektovo orientovaný framework pre jazyk C postavený na GLib. GObject je navrhnutý tak aby bolo možné exportovať jeho C-API\footnote{API - rozhranie pre programovanie aplikácií} pre ostatné jazyky za pomoci jazykových väzieb. Súčasťou GLib je dynamický typový systém postavený na štruktúre GType. Táto štruktúra v skratke definuje veľkosť triedy, inicializačné a deštrukčné funkcie, veľkost inštancie triedy, pravidlá inštancií\footnote{pre C++ typ nového operátora}, kopírovacie funkcie atď. Knižnica GObject poskytuje základnú objektovú triedu GObject. GObject neumožnuje viacnásobnú dedičnost a je implementovaný na spôsob Javových rozhraní. Každá GObject trieda sa implementuje pomocou minimálne dvoch štruktúr, ale zvykom je použit minimálne tri --- štruktúru triedy, štruktúru inštancií a privátnu štruktúru inštancií --- kedže jazyk C nemá prístupové modifikátory \verb|public|, \verb|protected| alebo \verb|private|. GObject definuje určitú formu kódu.
Pre hlavičkové súbory: %linky!
\begin{tiny}
\begin{verbatim}
/*
 * Copyright a licenčné informácie
 */
 
 /*Kontrola inklúzie.*/
 #ifndef _EV_PRESENTATION_TIMER_H_
 #define _EV_PRESENTATION_TIMER_H_
 
 /*Špeciálne makro, ktoré v prípade c++ kompilátora pridá extern okolo hlavičky.*/
 G_BEGIN_DECLS
 
 typedef struct _EvPresentationTimerClass        EvPresentationTimerClass;
 typedef struct _EvPresentationTimer             EvPresentationTimer;
 typedef struct _EvPresentationTimerPrivate      EvPresentationTimerPrivate;
 
 /* Typové makrá */
 #define EV_TYPE_PRESENTATION_TIMER              (ev_presentation_timer_get_type ())
 #define EV_PRESENTATION_TIMER(object)           (G_TYPE_CHECK_INSTANCE_CAST ((object), \\
                                                  EV_TYPE_PRESENTATION_TIMER, EvPresentationTimer))
 #define EV_PRESENTATION_TIMER_CLASS(klass)      (G_TYPE_CHECK_CLASS_CAST ((klass), \\
                                                  EV_TYPE_PRESENTATION_TIMER, EvPresentationTimerClass))
 #define EV_IS_PRESENTATION_TIMER(object)        (G_TYPE_CHECK_INSTANCE_TYPE ((object), \\
                                                  EV_TYPE_PRESENTATION_TIMER))
 #define EV_IS_PRESENTATION_TIMER_CLASS(klass)   (G_TYPE_CHECK_CLASS_TYPE ((klass), \\
                                                  EV_TYPE_PRESENTATION_TIMER))
 #define EV_PRESENTATION_TIMER_GET_CLASS(object) (G_TYPE_INSTANCE_GET_CLASS ((object), \\
                                                  EV_TYPE_PRESENTATION_TIMER, \\
                                                  EvPresentationTimerClass))
 
 struct _EvPresentationTimerClass
 {
         GtkDrawingAreaClass parent_class;
         /* členovia triedy */
 };
 
 struct _EvPresentationTimer
 {
         GtkDrawingArea                  parent_instance;
         /*privátna štruktúra je potom definovaná v .c súbore*/
         EvPresentationTimerPrivate     *priv;
         /*členovia inštancie*/
 };
 
 void                    ev_presentation_timer_set_pages         (EvPresentationTimer *ev_timer,
                                                                  guint pages);
 void                    ev_presentation_timer_set_page          (EvPresentationTimer *ev_timer,
                                                                  guint page);
 void                    ev_presentation_timer_start             (EvPresentationTimer *ev_timer);
 void                    ev_presentation_timer_stop              (EvPresentationTimer *ev_timer);
 void                    ev_presentation_timer_set_time          (EvPresentationTimer *ev_timer,
                                                                  gint time);
 GType                   ev_presentation_timer_get_type          (void);
 GtkWidget              *ev_presentation_timer_new               (void);
 
 G_END_DECLS
 
 #endif /*
\end{verbatim}
\end{tiny} %http://developer.gnome.org/gobject/unstable/gobject-Type-Information.html#g-type-class-add-private
Pri inicializácii triedy \verb|ev_presentation_timer_class_init()| musí funkcia \verb|g_type_class_add_private()| registrovat privátnu štruktúru pre triedu. Pri alokácii objektu sú privátne štruktúry pre typy a typy ich rodičov sprístupnené sekvenčne v rovnakom pamäťovom bloku ako ich verejné štruktúry. Nakoniec pri inicializácii inštancie v tomto prípade \verb|ev_presentation_timer_init()| sa za pomoci makra \verb|G_TYPE_INSTANCE_GET_PRIVATE()| získa ukazateľ na privátnu dátovú štruktúru. Na implementáciu \\
\verb|ev_presentation_timer_get_type()| funkcie a na definíciu ukazateľa na rodičovskú triedu sa používa makro \verb|G_DEFINE_TYPE|.
%properties

\section{GTK+}
\section{Cairo}

\bibliographystyle{plain} % sets plain bibliography style  
\bibliography{bib-db}     % BibTeX database file  

\end{document} 
