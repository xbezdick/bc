\documentclass[12pt,oneside,draft]{fithesis2}  
\usepackage[slovak]{babel}
\usepackage[utf8]{inputenc}
\usepackage[T1]{fontenc}  
\usepackage[plainpages=false,pdfpagelabels,unicode]{hyperref}  
 
\thesistitle{Prezentační režim pro PDF prohlížeč \emph{evince}}
\thesissubtitle{Bakalárska Práca}  
\thesisstudent{Lukáš Bezdička}  
\thesiswoman{false}  
\thesisfaculty{fi}  
\thesisyear{2011}  
\thesisadvisor{RNDr. Jan Kasprzak}  
\thesislang{sk}  
 
\begin{document}  
\FrontMatter  
\ThesisTitlePage  
 
\begin{ThesisDeclaration}  
\DeclarationText  
\AdvisorName  
\end{ThesisDeclaration}  
 
\begin{ThesisThanks}  
Rád by som poďakoval vývojárom \emph{evince} za ustretovú pomoc pri písaní práce a obzvlášť José Aliste.
\end{ThesisThanks}  
 
\begin{ThesisAbstract}  
...
\end{ThesisAbstract}  
 
\begin{ThesisKeyWords}  
keyword1, keyword2, etc.  
\end{ThesisKeyWords}  
 
\MainMatter  
%\sloppy
\chapter{Úvod}  
This is the first chapter of the thesis.   
...  
 
\tableofcontents          % prints table of contents  
 
\chapter{Introduction}    % first chapter followed by  
                          % many others  

 
\chapter{GNOME Platforma}
V roku 1997 Spencer Kimball a Peter Mattis vytvorili projekt GTK+ pre projekt GIMP. V tom istom roku Miguel de Icaza a Federico Mena založili projekt GNOME, ktorého cieľom je vytvorit jednoduché prostredie pracovnej plochy. Tieto projekty zostali na seba silne naviazané a spoločne s ďaľšími tvoria platformu GNOME. V tabuľke \ref{tab.GNOME} je prehľad GNOME platformy.
\begin{table}[h]
\begin{center}
\begin{tabular}{|c||c||c||c||c||c||c|}
\hline \multicolumn{3}{|c||}{\begin{tiny}
\textit{Užívateľské prostredie}
\end{tiny}} & \begin{tiny}
\textit{Multimédiá}
\end{tiny} & \begin{tiny}
\textit{Komunikácia}
\end{tiny} & \begin{tiny}
\textit{Ukladanie dát}
\end{tiny} & \begin{tiny}
\textit{Utility}
\end{tiny}\\
\multicolumn {1}{|c}{GTK+} & \multicolumn {1} {c} {Cairo} & \multicolumn {1} {c||} {Clutter} & GStreamer & Telepathy & EDS & Champlain \\
\multicolumn{1}{|c}{ATK} & \multicolumn{1}{c}{Pango} & \multicolumn{1}{c||}{Webkit} & Canberra & Avahi & GDA & Enchant \\ \cline{1-3}
\multicolumn{3}{|c||}{\begin{tiny}
\textit{Jadro}
\end{tiny}} & Pulseaudio &GUPnP & Tracker & Poppler \\
\multicolumn{1}{|c}{GIO} & \multicolumn{1}{c}{Glib} & \multicolumn{1}{c||}{GOBject} & & & GeoClue \\ \hline \hline
\multicolumn{3}{|c||}{\begin{tiny}
\textit{Systémová integrácia}
\end{tiny}} & \multicolumn{4}{|c|}{{\tiny \textit{Integrácia do pracovného prostredia}}} \\
\multicolumn{1}{|c}{upower} & \multicolumn{1}{c}{udisks} & \multicolumn{1}{c||}{policykit} & \multicolumn{1}{|c}{packagekit} & \multicolumn{1}{c}{libnotify} & \multicolumn{2}{c|}{seahorse} \\
\hline 
\end{tabular}
\caption{Platforma GNOME}
\label{tab.GNOME}
\end{center}
\end{table}

\section{Glib}
\bibliographystyle{plain} % sets plain bibliography style  
\bibliography{bib-db}     % BibTeX database file  
 
\end{document} 
