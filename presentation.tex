% $Header: /cvsroot/latex-beamer/latex-beamer/solutions/conference-talks/conference-ornate-20min.en.tex,v 1.6 2004/10/07 20:53:08 tantau Exp $

\documentclass{beamer}
\setbeameroption{hide  notes}
\newcommand\uv[1]{\quotedblbase #1\textquotedblleft}%
% This file is a solution template for:

% - Talk at a conference/colloquium.
% - Talk length is about 20min.
% - Style is ornate.



% Copyright 2004 by Till Tantau <tantau@users.sourceforge.net>.
%
% In principle, this file can be redistributed and/or modified under
% the terms of the GNU Public License, version 2.
%
% However, this file is supposed to be a template to be modified
% for your own needs. For this reason, if you use this file as a
% template and not specifically distribute it as part of a another
% package/program, I grant the extra permission to freely copy and
% modify this file as you see fit and even to delete this copyright
% notice. 


\mode<presentation>
{
  \usetheme{Warsaw}
  % or ...

  \setbeamercovered{transparent}
  % or whatever (possibly just delete it)
}


%\usepackage[english]{babel}
\usepackage[slovak]{babel}
% or whatever

\usepackage[utf8]{inputenc}
% or whatever

\usepackage{times}
\usepackage[T1]{fontenc}
% Or whatever. Note that the encoding and the font should match. If T1
% does not look nice, try deleting the line with the fontenc.


\title[Prezentačný mód evince] % (optional, use only with long paper titles)
{Prezentačný režim pre PDF prehliadač \emph{evince}}

\author[Bezdička L.] % (optional, use only with lots of authors)
{Lukáš Bezdička\inst{1}}
% - Give the names in the same order as the appear in the paper.
% - Use the \inst{?} command only if the authors have different
%   affiliation.

\institute[FI MUNI] % (optional, but mostly needed)
{
  \inst{1}%
  Fakulta Informatiky\\
  Masarykova Univerzita
}
% - Use the \inst command only if there are several affiliations.
% - Keep it simple, no one is interested in your street address.

\date[Jar 2011] % (optional, should be abbreviation of conference name)
{Obhajoba úvodu k bakalárskej práci}
% - Either use conference name or its abbreviation.
% - Not really informative to the audience, more for people (including
%   yourself) who are reading the slides online

\subject{Prezentačný mód evince}
% This is only inserted into the PDF information catalog. Can be left
% out. 



% If you have a file called "university-logo-filename.xxx", where xxx
% is a graphic format that can be processed by latex or pdflatex,
% resp., then you can add a logo as follows:

% \pgfdeclareimage[height=0.5cm]{university-logo}{university-logo-filename}
% \logo{\pgfuseimage{university-logo}}



% Delete this, if you do not want the table of contents to pop up at
% the beginning of each subsection:
%\AtBeginSubsection[]
%{
%  \begin{frame}<beamer>
%    \frametitle{Outline}
%    \tableofcontents[currentsection,currentsubsection]
%  \end{frame}
%}


% If you wish to uncover everything in a step-wise fashion, uncomment
% the following command: 

%\beamerdefaultoverlayspecification{<+->}


\begin{document}

\begin{frame}
  \titlepage
\end{frame}

%\begin{frame}
%  \frametitle{Outline}
%  \tableofcontents
%  % You might wish to add the option [pausesections]
%\end{frame}


% Structuring a talk is a difficult task and the following structure
% may not be suitable. Here are some rules that apply for this
% solution: 

% - Exactly two or three sections (other than the summary).
% - At *most* three subsections per section.
% - Talk about 30s to 2min per frame. So there should be between about
%   15 and 30 frames, all told.

% - A conference audience is likely to know very little of what you
%   are going to talk about. So *simplify*!
% - In a 20min talk, getting the main ideas across is hard
%   enough. Leave out details, even if it means being less precise than
%   you think necessary.
% - If you omit details that are vital to the proof/implementation,
%   just say so once. Everybody will be happy with that.

\section{Motivácia}

\subsection{Prezentácie na viacerých monitoroch}

\begin{frame}
  \frametitle{Momentálny stav a motivácia}
%\framesubtitle{}
  % - A title should summarize the slide in an understandable fashion
  %   for anyone how does not follow everything on the slide itself.

  \begin{itemize}
  \item
    Veľa programov na tvorbu prezentácií a ešte viac formátov. 
    \pause
  \item
    LaTeX Beamer na tvorbu prezentácií vo formáte PDF. 
    \pause
  \item
    Nulová podpora viacerých monitorov vo väčšine PDF prehliadačov. 
    \pause
  \item
    Rozhodnutie upraviť \emph{evince}.
  \end{itemize}
\end{frame}

\subsection{Evince}
\begin{frame}
\frametitle{Prečo evince?}
Ide o pravdepodobne najpoužívanejší prehliadač PDF dokumentov s otvoreným kódom.
\end{frame}
\begin{frame}
\emph{Evince} vznikol ako projekt zjednocujúci prehliadače dokumentov v prostredí GNOME do jednej aplikácie.
Pozostáva z:

\begin{description}
\item [backend] --- prípojné body pre knižnice na prácu s jednotlivými formátmi.
\item [libdocument] --- slúži na abstrakciu nad dokumentmi.
\item [libview] --- knižnica určená na samotné zobrazenie.
\item [shell] --- samotné jadro programu.
\end{description}

\end{frame}

\section{Práca}
\subsection{Úprava programu evince}
\begin{frame}
Práca pozostávala z:
\begin{itemize}
\item Vytvorenia kontrolného okna.
\item Určenia geometrie monitorov a presunutia okien.
\item Vytvorenia ovládacieho prvku (tzv. widget) časovača a jeho umiestnenia do kontrolného okna.
\end{itemize}
\end{frame}

\section{Ukážka}

\subsection{Výsledok?}

\begin{frame}
\frametitle{Praktická ukážka}
\note{
\begin{description}
\item Toto sú poznámky!
\item Na ľavo je \uv{sidebar} s náhľadmi.
\item Dole je časovač --- ako vidíte dochádza mi čas.
\end{description}}
  By mal vyzerať asi takto\ldots
\end{frame}


 
\section{Zhrnutie}
\newcount\opaqueness
\newdimen\offset
\begin{frame}

  \frametitle<presentation>{Zhrnutie}
\animate<2-5>
% Actual animation values. Try <1-31>
\begin{itemize}
\item[]
\animatevalue<1-5>{\opaqueness}{0}{100}%
\animatevalue<1-5>{\offset}{6cm}{0cm}%
\begin{colormixin}{\the\opaqueness!averagebackgroundcolor}
\hspace{\offset} Aj animácie fungujú. {\color{olive} Ďakujem}!
\end{colormixin}
\end{itemize}
\end{frame}

\end{document}


