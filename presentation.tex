\documentclass{beamer}
\setbeameroption{hide notes}
\newcommand\uv[1]{\quotedblbase #1\textquotedblleft}%

\mode<presentation>
{
  \usetheme{Warsaw}
  \setbeamercovered{transparent}
}
\usepackage[slovak]{babel}
\usepackage[utf8]{inputenc}
\usepackage{times}
\usepackage[T1]{fontenc}
\title[Prezentačný mód evince]
{Prezentačný režim pre PDF prehliadač \emph{evince}}
\author[Bezdička L.]
{Lukáš Bezdička}
\institute[FI MUNI]
{
  Fakulta informatiky\\
  Masarykova univerzita
}
\date[Jar 2011]
{Bakalárska práca}
\subject{Prezentačný mód evince}

\pgfdeclareimage[height=0.5cm]{university-logo}{fi-logo}
\logo{\pgfuseimage{university-logo}}

% If you wish to uncover everything in a step-wise fashion, uncomment
% the following command: 

%\beamerdefaultoverlayspecification{<+->}


\begin{document}

\begin{frame}
  \titlepage
  \note{
\begin{itemize}
\item Pekne sa pozdrav. 
\item Napríklad: Milé dámy a páni.
\item NEPOZERAJ SA PRÍLIŠ ČASTO NA POZNÁMKY!
\end{itemize}
}
\end{frame}

\section{Motivácia}

\subsection{Prezentácie na viacerých monitoroch}

\begin{frame}
  \note{Hlavne klud}
  \frametitle{Momentálny stav a motivácia}

  \begin{itemize}
  \item
    Veľa programov na tvorbu prezentácií a ešte viac formátov. 
    \pause
  \item
    LaTeX Beamer na tvorbu prezentácií vo formáte PDF. 
    \pause
  \item
    Nulová podpora viacerých monitorov vo väčšine PDF prehliadačov. 
    \pause
  \item
    Rozhodnutie upraviť \emph{evince}.
  \end{itemize}
\end{frame}

\subsection{Evince}
\begin{frame}
\frametitle{Prečo evince?}
\note{preco nie}
Ide o pravdepodobne najpoužívanejší prehliadač PDF dokumentov s otvoreným kódom.
\end{frame}
\begin{frame}
\note{neobkecavat nezasekavat sa}
\emph{Evince} vznikol ako projekt zjednocujúci prehliadače dokumentov v prostredí GNOME do jednej aplikácie.
Pozostáva z:

\begin{description}
\item [backend] --- prípojné body pre knižnice na prácu s jednotlivými formátmi.
\item [libdocument] --- slúži na abstrakciu nad dokumentmi.
\item [libview] --- knižnica určená na zobrazenie.
\item [shell] --- jadro programu.
\end{description}

\end{frame}

\section{Práca}
\subsection{Úprava programu evince}
\begin{frame}
\note{widget! daj si pozor}
Moja práca:
\begin{itemize}
\item Vytvorenie kontrolného okna.
\item Určenie geometrie monitorov a presunutie okien.
\item Vytvorenie ovládacieho prvku (tzv. widget) časovača a jeho umiestnenie do kontrolného okna.
\end{itemize}
\end{frame}

\section{Ukážka}

\subsection{Výsledok?}

\begin{frame}
\frametitle{Praktická ukážka}
\note{
\begin{description}
\item Toto sú poznámky!
\item Na ľavo je \uv{sidebar} s náhľadmi.
\item Dole je časovač --- ako vidíte, dochádza mi čas.
\end{description}}
  By mal vyzerať asi takto\ldots
\end{frame}


 
\section{Záver}
\newcount\opaqueness
\newdimen\offset
\begin{frame}

  \frametitle<presentation>{Zhrnutie}
\animate<2-5>
% Actual animation values. Try <1-31>
\begin{itemize}
\item[]
\animatevalue<1-5>{\opaqueness}{0}{100}%
\animatevalue<1-5>{\offset}{6cm}{0cm}%
\begin{colormixin}{\the\opaqueness!averagebackgroundcolor}
\hspace{\offset} Aj animácie fungujú. {\color{olive} Ďakujem}!
\end{colormixin}
\end{itemize}
\end{frame}

\end{document}


